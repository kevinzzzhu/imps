## User persona:
- **Background & Experience:** 19-year-old student who plays piano and guitar casually. No programming knowledge and limited exposure to music tech.
- **Goals:** Have fun jamming with an AI accompanist, practice improvisation at home, and create interesting sounds without needing a band. Alicia wants to improve her skills and confidence by experimenting in a safe, low-pressure environment.
- **Frustrations:** Intimidated by complex software or technical jargon. She gets frustrated if setup is cumbersome or if she must read manuals. Loses interest quickly if the interface is confusing or if she fears “breaking something.”
- **Interaction Style:** Hands-on and visual. Prefers plug-and-play experiences – she’d rather **start playing immediately** than tweak settings. Likely to stick with default options and use features that are presented in an obvious, guided way. Needs gentle guidance (tooltips, tutorials) to discover new features.

## Scenarios
###  User activities
- **Starting with IMPSY**
- **Jamming with AI**
- **Learning the basics**

###  User tasks
- **For starting with IMPSY:**
    -   Launch and onboard quickly.
    -   Start a new session with ease.
    -   Connect instrument and understand basic controls intuitively.
- **For jamming with AI:**
    -   Engage in call-and-response with AI.
    -   Experience supportive improvisation.
    -   Enjoy a seamless musical jam.
- **For learning the basics:**
    -   Learn main controls via guidance.
    -   Explore default features comfortably.
    -   Easily discover how to end session.

## User stories
###  Release 1: Core Jamming & Immediate Use
- Onboarding: Simple first-time user tutorial.
- Quick Start: Begin jamming via minimal steps.
- Input: Easy instrument connection/detection.
- Core AI: AI plays when user pauses.
- Interplay: Natural call-and-response.
- No Setup: Avoids technical configurations.
- Prompts: Clear guidance for first actions.
- Defaults: Sensible default settings applied.
- Confidence: Safe environment to start playing.
- Basic Controls: Obvious way to start/stop AI.

###  Release 2: Enhanced Practice & User Comfort
- AI Status: Visual cue for AI activity (listening/playing).
- Help: Contextual tooltips for UI elements.
- Interface: Unintimidating and friendly design.
- Practice Aid: AI responses aid improvisation.
- Sound Quality: Pleasing default AI sounds.
- Discovery: Basic features easily found.
- Feedback: Positive, encouraging interaction flow.
- Environment: Low-pressure practice zone.
- Guidance: Gentle prompts for next steps.

###  Release 3: Early Exploration & Sustained Engagement
- Tutorial Access: Revisit onboarding/tutorial easily.
- AI Variety: Simple choice of AI "styles" or "moods".
- Engagement: Encourage repeated use for practice.
- Fun Factor: Focus on joyful sound creation.
- Skill Building: Support confidence growth over time.
- Session End: Simple way to conclude a session.
- Exploration: Discover new sounds without complexity.
- No Frustration: Avoid confusing options or jargon.
- Creative Spark: AI helps create interesting musical ideas. 