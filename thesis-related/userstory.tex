## User stories

- **As a beginner musician**, I want to **jam with an AI that plays along when I stop** so that I can practice improvisation without feeling intimidated. (The system should seamlessly take over in moments when I pause, creating a call-and-response interplay, without requiring any technical setup from me.)
- **As a first-time user**, I want an **easy onboarding** with clear prompts and default settings so that I can start making music with IMPSY right away. (For example, a quick tutorial and auto-detected instrument input should let me begin playing in minutes, no coding or complex configuration needed.)
- **As an intermediate musician**, I want to **train a new AI model after my performance** so that the system learns my style over time. (After a jam session, I should be able to save the session log and initiate training of a personalized model using that data, without dealing with code – just one-click to “learn from my last session.”)
- **As a returning user**, I want to **access and manage my recent projects** so that I can seamlessly continue where I left off. (I expect a dashboard showing my last few sessions/models, with options to load a project, rename it, or start a new one, ensuring my work is organized and easily accessible.)
- **As a musician with some tech experience**, I want to **visualize the performance logs and AI’s state** so that I can understand what the AI learned or is doing. (For instance, seeing a timeline of my input vs. the AI’s output, or graphs of the training progress/loss after I update the model, to build trust and insight into the system’s behavior.)
- **As an advanced user**, I want the ability to **tweak advanced settings (e.g. model size or hyperparameters)** so that I can fine-tune the AI’s performance for my needs. (The interface should offer an “Advanced Settings” section – hidden by default – where I can adjust things like model complexity, training iterations, or call-response mode settings. This caters to power users without cluttering the experience for everyone else.)
- **As a live performer**, I need the system to be **stable and responsive in real-time** so that it augments my show without hiccups. (During a gig, IMPSY should react immediately when I stop playing and produce musically sensible output. The interface should provide a performance mode with a clean layout – e.g. big buttons for start/stop AI, and indicators of what the AI is doing – to use on stage confidently.)

## Personas

To design an intuitive experience, we identified key personas representing our user base. Each persona has different experience levels, goals, and interaction styles, ensuring the interface accommodates a wide spectrum of musicians:

### Persona 1: **Alicia – The Aspiring Hobbyist** (Beginner Musician)

- **Background & Experience:** 19-year-old student who plays piano and guitar casually. No programming knowledge and limited exposure to music tech.
- **Goals:** Have fun jamming with an AI accompanist, practice improvisation at home, and create interesting sounds without needing a band. Alicia wants to improve her skills and confidence by experimenting in a safe, low-pressure environment.
- **Frustrations:** Intimidated by complex software or technical jargon. She gets frustrated if setup is cumbersome or if she must read manuals. Loses interest quickly if the interface is confusing or if she fears “breaking something.”
- **Interaction Style:** Hands-on and visual. Prefers plug-and-play experiences – she’d rather **start playing immediately** than tweak settings. Likely to stick with default options and use features that are presented in an obvious, guided way. Needs gentle guidance (tooltips, tutorials) to discover new features.

### Persona 2: **Ben – The Experienced Performer** (Intermediate Musician)

- **Background & Experience:** thirty-something semi-professional musician (e.g. a keyboardist in a local band). Comfortable with common music software (DAWs, synthesizer presets) but not a coder. Has a good grasp of musical concepts and some tech savvy with gear.
- **Goals:** Use IMPSY to **enhance practice sessions and songwriting** – for instance, have the AI riff off his solos to spark new ideas. Possibly integrate the system in live jam sessions or teaching (showing students how AI can be a creative partner). Wants to save sessions and refine the AI over time to better fit his style.
- **Frustrations:** Gets impatient with rigid tools that don’t allow any customization of the music output. If the AI’s responses feel repetitive or off-style, he wants a way to improve it (e.g. by training with more of his own playing). Also dislikes when projects get disorganized – he values being able to manage and reload saved models or recordings easily.
- **Interaction Style:** Exploratory and iterative. Ben will dive a bit deeper into settings than a novice: for example, he’ll explore menu options to train the model on his latest jam, or switch the AI’s instrument sound. However, he still expects **a friendly UI** – he appreciates visual feedback (like graphs of AI learning or live indicators of tempo/beat) and will use moderately advanced features if they are well-explained. He might consult the user guide for specific tasks, but generally prefers learning by doing.

### Persona 3: **Casey – The Tech-Savvy Music Creator** (Expert/Power User)

- **Background & Experience:** 40-year-old electronic music producer and tinkerer. Has experience with programming and music technology (e.g. builds DIY instruments, familiar with tools like Max/MSP or machine learning libraries for music). Possibly an academic or hobbyist who loves experimenting with AI in music.
- **Goals:** Push the boundaries of the IMPSY system – e.g., fine-tune the AI’s neural network settings, integrate IMPSY with custom hardware, or use it in avant-garde performances. Casey wants **control and flexibility**, aiming to create unique musical experiences. They might even extend the system by training different models or chaining it with other software.
- **Frustrations:** Feels constrained if the interface oversimplifies everything or hides too much. If advanced options are not available, they might get frustrated that they can’t adjust the model’s behavior (e.g., the hyperparameters, the call-response vs. simultaneous mode). Also critical of performance issues – expects the system to handle long sessions, complex inputs, and provide debugging info if something goes wrong.
- **Interaction Style:** Methodical and willing to navigate complexity. Casey will actively seek out the “Advanced” section of the UI and read documentation on how IMPSY works under the hood. They might import/export data or models, and use the logging and visualization tools extensively to analyze performance. However, they still benefit from a well-structured UI: e.g., they appreciate that advanced features are accessible but **not in the way** when not needed (so the workflow is efficient). Casey might use keyboard shortcuts or script certain interactions if supported, showing a high level of engagement with the system.