## User persona:
- **Background & Experience:** 40-year-old electronic music producer and tinkerer. Has experience with programming and music technology (e.g. builds DIY instruments, familiar with tools like Max/MSP or machine learning libraries for music). Possibly an academic or hobbyist who loves experimenting with AI in music.
- **Goals:** Push the boundaries of the IMPSY system – e.g., fine-tune the AI’s neural network settings, integrate IMPSY with custom hardware, or use it in avant-garde performances. Casey wants **control and flexibility**, aiming to create unique musical experiences. They might even extend the system by training different models or chaining it with other software.
- **Frustrations:** Feels constrained if the interface oversimplifies everything or hides too much. If advanced options are not available, they might get frustrated that they can’t adjust the model’s behavior (e.g., the hyperparameters, the call-response vs. simultaneous mode). Also critical of performance issues – expects the system to handle long sessions, complex inputs, and provide debugging info if something goes wrong.
- **Interaction Style:** Methodical and willing to navigate complexity. Casey will actively seek out the “Advanced” section of the UI and read documentation on how IMPSY works under the hood. They might import/export data or models, and use the logging and visualization tools extensively to analyze performance. However, they still benefit from a well-structured UI: e.g., they appreciate that advanced features are accessible but **not in the way** when not needed (so the workflow is efficient). Casey might use keyboard shortcuts or script certain interactions if supported, showing a high level of engagement with the system.

## Scenarios
###  User activities
- **Deep System Customisation**
- **Advanced Performance & Integration**
- **Data Analysis & System Understanding**

###  User tasks
- **For deep system customisation:**
    -   Access and modify advanced AI model parameters (e.g., model size, hyperparameters).
    -   Adjust core system behaviours (e.g., call-response logic, training iterations).
    -   Manage different AI models and configurations.
- **For advanced performance & integration:**
    -   Utilise system for long, complex live performances reliably.
    -   Integrate IMPSY with custom hardware or other software.
    -   Explore options for scripting or extending system capabilities.
- **For data analysis & system understanding:**
    -   Access detailed performance logs and AI state information.
    -   Visualise training progress and model behaviour in depth.
    -   Import/export data or models for external analysis/use.
    -   Utilise debugging information for troubleshooting.

## User stories
###  Release 1: Foundation for Advanced Control & Data Access
- Advanced Settings Panel: Access to a dedicated (hidden by default) advanced settings section.
- Basic Hyperparameter Tweaks: Ability to adjust a few key training/model parameters.
- Performance Logging: Exportable detailed logs of user input and AI output.
- Model Management: Basic ability to save and load different trained model configurations.
- Stability for Long Use: System remains stable during extended sessions.
- Clear Documentation Access: Easy to find documentation on advanced features and system architecture.

###  Release 2: Enhanced Customisation & Visualisation
- Granular Model Control: More comprehensive set of adjustable hyperparameters.
- Training Configuration: Adjust training iterations and other learning parameters.
- Advanced Visualisation: Detailed graphs for training (loss, accuracy) and AI state.
- Mode Switching: Option to change core AI interaction modes (e.g., call-response vs. simultaneous play).
- Data Export: Export session data/logs in standard formats (e.g., MIDI, CSV).
- API/Integration Hooks: Basic API or hooks for potential external software integration.

###  Release 3: Full System Mastery & Extensibility
- Full Parameter Access: Control over nearly all configurable aspects of the AI model and system.
- Custom Model Training: Ability to train significantly different types of models if applicable.
- Scripting Interface: Support for scripting common tasks or interactions.
- Hardware Integration Support: Clear pathways/documentation for custom hardware input/output.
- Debugging Tools: Access to more detailed debugging information or modes.
- Advanced Data I/O: Import/export models or datasets in common ML formats.
- Extensibility Support: Framework or documentation for user-led system extensions.
- Resource Management: Tools to monitor and manage system resource usage during intensive tasks. 